\documentclass[10pt,a4paper]{article}

% Packages
\usepackage{fancyhdr}           % For header and footer
\usepackage{multicol}           % Allows multicols in tables
\usepackage{tabularx}           % Intelligent column widths
\usepackage{tabulary}           % Used in header and footer
\usepackage{hhline}             % Border under tables
\usepackage{graphicx}           % For images
\usepackage{xcolor}             % For hex colours
%\usepackage[utf8x]{inputenc}    % For unicode character support
\usepackage[T1]{fontenc}        % Without this we get weird character replacements
\usepackage{colortbl}           % For coloured tables
% \usepackage{setspace}           % For line height
\usepackage{lastpage}           % Needed for total page number
\usepackage{seqsplit}           % Splits long words.
%\usepackage{opensans}          % Can't make this work so far. Shame. Would be lovely.
\usepackage[normalem]{ulem}     % For underlining links
% Most of the following are not required for the majority
% of cheat sheets but are needed for some symbol support.
\usepackage{amsmath}            % Symbols
\usepackage{MnSymbol}           % Symbols
% \usepackage{wasysym}            % Symbols
%\usepackage[english,german,french,spanish,italian]{babel}              % Languages

% Document Info
\author{Greg Finzer (GregFinzer)}
% TODO: fix latex pdfinfo not working
% \pdfinfo{
%   /Title (c-naming-conventions.pdf)
%   /Creator (Cheatography)
%   /Author (Greg Finzer (GregFinzer))
%   /Subject (C Naming Conventions Cheat Sheet)
% }

% Lengths and widths
\addtolength{\textwidth}{6cm}
\addtolength{\textheight}{-1cm}
\addtolength{\hoffset}{-3cm}
\addtolength{\voffset}{-2cm}
\setlength{\tabcolsep}{0.2cm} % Space between columns
\setlength{\headsep}{-12pt} % Reduce space between header and content
\setlength{\headheight}{85pt} % If less, LaTeX automatically increases it
\renewcommand{\footrulewidth}{0pt} % Remove footer line
\renewcommand{\headrulewidth}{0pt} % Remove header line
\renewcommand{\seqinsert}{\ifmmode\allowbreak\else\-\fi} % Hyphens in seqsplit
% This two commands together give roughly
% the right line height in the tables
\renewcommand{\arraystretch}{1.3}
% TODO: fix latex setspace not working
% \onehalfspacing

% Commands
\newcommand{\SetRowColor}[1]{\noalign{\gdef\RowColorName{#1}}\rowcolor{\RowColorName}} % Shortcut for row colour
\newcommand{\mymulticolumn}[3]{\multicolumn{#1}{>{\columncolor{\RowColorName}}#2}{#3}} % For coloured multi-cols
\newcolumntype{x}[1]{>{\raggedright}p{#1}} % New column types for ragged-right paragraph columns
\newcommand{\tn}{\tabularnewline} % Required as custom column type in use

% Font and Colours
\definecolor{HeadBackground}{HTML}{333333}
\definecolor{FootBackground}{HTML}{666666}
\definecolor{TextColor}{HTML}{333333}
\definecolor{DarkBackground}{HTML}{007ACC}
\definecolor{LightBackground}{HTML}{EFF6FB}
\renewcommand{\familydefault}{\sfdefault}
\color{TextColor}

% Header and Footer
\pagestyle{fancy}
\fancyhead{} % Set header to blank
\fancyfoot{} % Set footer to blank
\fancyhead[L]{
\noindent
\begin{multicols}{3}
\begin{tabulary}{5.8cm}{C}
    \SetRowColor{DarkBackground}
    \vspace{-7pt}
    {\parbox{\dimexpr\textwidth-2\fboxsep\relax}{\noindent
        \hspace*{-6pt}}
    }
\end{tabulary}
\columnbreak
\begin{tabulary}{11cm}{L}
    \vspace{-2pt}\large{\bf{\textcolor{DarkBackground}{\textrm{C\# Naming Conventions Cheat Sheet}}}} \\
    \normalsize{by \textcolor{DarkBackground}{Greg Finzer (GregFinzer)} via \textcolor{DarkBackground}{}}
\end{tabulary}
\end{multicols}}

\fancyfoot[L]{ \footnotesize
\noindent
\begin{multicols}{3}
\begin{tabulary}{5.8cm}{LL}
  \SetRowColor{FootBackground}
  \mymulticolumn{2}{p{5.377cm}}{\bf\textcolor{white}{Cheatographer}}  \\
  \vspace{-2pt}Greg Finzer (GregFinzer) \\
  \uline{cheatography.com/gregfinzer} \\
        \uline{\seqsplit{www.kellermansoftware.com}}
  \end{tabulary}
\vfill
\columnbreak
\begin{tabulary}{5.8cm}{L}
  \SetRowColor{FootBackground}
  \mymulticolumn{1}{p{5.377cm}}{\bf\textcolor{white}{Cheat Sheet}}  \\
   \vspace{-2pt}Published 15th October, 2018.\\
   Updated 15th October, 2018.\\
   Page {\thepage} of \pageref{LastPage}.
\end{tabulary}
\vfill
\columnbreak
\begin{tabulary}{5.8cm}{L}
  \SetRowColor{FootBackground}
  \mymulticolumn{1}{p{5.377cm}}{\bf\textcolor{white}{Sponsor}}  \\
  \SetRowColor{white}
  \vspace{-5pt}
  %\includegraphics[width=48px,height=48px]{dave.jpeg}
  Measure your website readability!\\
  www.readability-score.com
\end{tabulary}
\end{multicols}}




\begin{document}
\raggedright
\raggedcolumns

% Set font size to small. Switch to any value
% from this page to resize cheat sheet text:
% www.emerson.emory.edu/services/latex/latex_169.html
\footnotesize % Small font.

\begin{multicols*}{2}

\begin{tabularx}{8.4cm}{x{3.116 cm} x{1.444 cm} x{3.04 cm} }
\SetRowColor{DarkBackground}
\mymulticolumn{3}{x{8.4cm}}{\bf\textcolor{white}{Casing Rules}}  \tn
% Row 0
\SetRowColor{LightBackground}
{\bf{Identifier}} & {\bf{Case}} & {\bf{Example}} \tn 
% Row Count 2 (+ 2)
% Row 1
\SetRowColor{white}
Namespace & Pascal & System.Drawing \tn 
% Row Count 3 (+ 1)
% Row 2
\SetRowColor{LightBackground}
Class & Pascal & Customer \tn 
% Row Count 4 (+ 1)
% Row 3
\SetRowColor{white}
Class Field & Camel & \_transferConfig \tn 
% Row Count 5 (+ 1)
% Row 4
\SetRowColor{LightBackground}
Interface & Pascal & IPerson \tn 
% Row Count 6 (+ 1)
% Row 5
\SetRowColor{white}
Property & Pascal & TransferConfig \tn 
% Row Count 7 (+ 1)
% Row 6
\SetRowColor{LightBackground}
Method & Pascal & TransferAccount \tn 
% Row Count 8 (+ 1)
% Row 7
\SetRowColor{white}
Parameter & Camel & accountNumber \tn 
% Row Count 9 (+ 1)
% Row 8
\SetRowColor{LightBackground}
Constant & Pascal & \seqsplit{InfiniteThrottle} \tn 
% Row Count 10 (+ 1)
% Row 9
\SetRowColor{white}
Enumeration Type & Pascal & BrowserType \tn 
% Row Count 11 (+ 1)
% Row 10
\SetRowColor{LightBackground}
Enumeration Value & Pascal & \seqsplit{InternetExplorer} \tn 
% Row Count 13 (+ 2)
% Row 11
\SetRowColor{white}
Event & Pascal & \seqsplit{TransferComplete} \tn 
% Row Count 14 (+ 1)
% Row 12
\SetRowColor{LightBackground}
Exception Class & Pascal & \seqsplit{TransferException} \tn 
% Row Count 16 (+ 2)
% Row 13
\SetRowColor{white}
Pre-Processor & Upper & NETSTANDARD \tn 
% Row Count 17 (+ 1)
\hhline{>{\arrayrulecolor{DarkBackground}}---}
\SetRowColor{LightBackground}
\mymulticolumn{3}{x{8.4cm}}{Except for parameters, class fields, and pre-processor directives, all other identifiers follow a Pascal naming convention.}  \tn 
\hhline{>{\arrayrulecolor{DarkBackground}}---}
\end{tabularx}
\par\addvspace{1.3em}

\begin{tabularx}{8.4cm}{X}
\SetRowColor{DarkBackground}
\mymulticolumn{1}{x{8.4cm}}{\bf\textcolor{white}{Namespaces}}  \tn
% Row 0
\SetRowColor{LightBackground}
\mymulticolumn{1}{x{8.4cm}}{Choose names that indicate functionality} \tn 
% Row Count 1 (+ 1)
% Row 1
\SetRowColor{white}
\mymulticolumn{1}{x{8.4cm}}{The general format for a namespace name is as follows:} \tn 
% Row Count 3 (+ 2)
% Row 2
\SetRowColor{LightBackground}
\mymulticolumn{1}{x{8.4cm}}{{\emph{\textless{}Company\textgreater{}.(\textless{}Project\textgreater{}){[}.\textless{}Feature\textgreater{}{]}{[}.\textless{}Subnamespace\textgreater{}{]}}}} \tn 
% Row Count 5 (+ 2)
% Row 3
\SetRowColor{white}
\mymulticolumn{1}{x{8.4cm}}{{\bf{Example:}}  NewtonSoft.Json.Linq} \tn 
% Row Count 6 (+ 1)
\hhline{>{\arrayrulecolor{DarkBackground}}-}
\end{tabularx}
\par\addvspace{1.3em}

\begin{tabularx}{8.4cm}{X}
\SetRowColor{DarkBackground}
\mymulticolumn{1}{x{8.4cm}}{\bf\textcolor{white}{Assemblies and DLL Names}}  \tn
% Row 0
\SetRowColor{LightBackground}
\mymulticolumn{1}{x{8.4cm}}{Choose names that suggest large chunks of functionality.} \tn 
% Row Count 2 (+ 2)
% Row 1
\SetRowColor{white}
\mymulticolumn{1}{x{8.4cm}}{It is advisable if assembly and DLL names follow the namespace names.} \tn 
% Row Count 4 (+ 2)
% Row 2
\SetRowColor{LightBackground}
\mymulticolumn{1}{x{8.4cm}}{The following pattern may be followed for naming DLLs:} \tn 
% Row Count 6 (+ 2)
% Row 3
\SetRowColor{white}
\mymulticolumn{1}{x{8.4cm}}{{\emph{\textless{}Company\textgreater{}.\textless{}Component\textgreater{}.dll}}} \tn 
% Row Count 7 (+ 1)
% Row 4
\SetRowColor{LightBackground}
\mymulticolumn{1}{x{8.4cm}}{Where {\emph{\textless{}component\textgreater{}}} contains one or more dot separated clauses.} \tn 
% Row Count 9 (+ 2)
% Row 5
\SetRowColor{white}
\mymulticolumn{1}{x{8.4cm}}{{\bf{Example}}: NewtonSoft.Json.dll} \tn 
% Row Count 10 (+ 1)
\hhline{>{\arrayrulecolor{DarkBackground}}-}
\end{tabularx}
\par\addvspace{1.3em}

\begin{tabularx}{8.4cm}{X}
\SetRowColor{DarkBackground}
\mymulticolumn{1}{x{8.4cm}}{\bf\textcolor{white}{Parameters}}  \tn
% Row 0
\SetRowColor{LightBackground}
\mymulticolumn{1}{x{8.4cm}}{Choose parameter names that indicate what data is being affected.} \tn 
% Row Count 2 (+ 2)
% Row 1
\SetRowColor{white}
\mymulticolumn{1}{x{8.4cm}}{{\bf{Good}}:  {\emph{firstName}} - Uses camel casing and is descriptive} \tn 
% Row Count 4 (+ 2)
% Row 2
\SetRowColor{LightBackground}
\mymulticolumn{1}{x{8.4cm}}{{\bf{Bad}}:  {\emph{decimalSalary}} - Name should not be based on type} \tn 
% Row Count 6 (+ 2)
\hhline{>{\arrayrulecolor{DarkBackground}}-}
\end{tabularx}
\par\addvspace{1.3em}

\begin{tabularx}{8.4cm}{X}
\SetRowColor{DarkBackground}
\mymulticolumn{1}{x{8.4cm}}{\bf\textcolor{white}{Resources}}  \tn
% Row 0
\SetRowColor{LightBackground}
\mymulticolumn{1}{x{8.4cm}}{Nested identifiers with clear hierarchy} \tn 
% Row Count 1 (+ 1)
% Row 1
\SetRowColor{white}
\mymulticolumn{1}{x{8.4cm}}{{\bf{Example}}: {\emph{Menus.File.Close.Text}}} \tn 
% Row Count 2 (+ 1)
\hhline{>{\arrayrulecolor{DarkBackground}}-}
\end{tabularx}
\par\addvspace{1.3em}

\begin{tabularx}{8.4cm}{X}
\SetRowColor{DarkBackground}
\mymulticolumn{1}{x{8.4cm}}{\bf\textcolor{white}{Classes, Structs, and Interfaces}}  \tn
% Row 0
\SetRowColor{LightBackground}
\mymulticolumn{1}{x{8.4cm}}{Use pascal cased nouns, noun phrases or adjective phrases like Customer or Invoice.  This distinguishes type names from methods, which are named with verb phrases like SaveCustomer or LoadInvoice.} \tn 
% Row Count 4 (+ 4)
% Row 1
\SetRowColor{white}
\mymulticolumn{1}{x{8.4cm}}{{\bf{Use of suffixes and prefixes}}} \tn 
% Row Count 5 (+ 1)
% Row 2
\SetRowColor{LightBackground}
\mymulticolumn{1}{x{8.4cm}}{Derived class should have suffix representing the base class. e,g OvalShape} \tn 
% Row Count 7 (+ 2)
% Row 3
\SetRowColor{white}
\mymulticolumn{1}{x{8.4cm}}{\seqsplit{TransferCompleteEventHandler} – EventHandler suffix for handlers} \tn 
% Row Count 9 (+ 2)
% Row 4
\SetRowColor{LightBackground}
\mymulticolumn{1}{x{8.4cm}}{TransferCompleteCallback – Callback suffix to delegates} \tn 
% Row Count 11 (+ 2)
% Row 5
\SetRowColor{white}
\mymulticolumn{1}{x{8.4cm}}{TransferException – Exception suffix for deriving from Exception} \tn 
% Row Count 13 (+ 2)
% Row 6
\SetRowColor{LightBackground}
\mymulticolumn{1}{x{8.4cm}}{AccountDictionary – Dictionary suffix for dictionary implementations} \tn 
% Row Count 15 (+ 2)
% Row 7
\SetRowColor{white}
\mymulticolumn{1}{x{8.4cm}}{SocketStream - Stream suffix for inheriting from System.IO.Stream} \tn 
% Row Count 17 (+ 2)
% Row 8
\SetRowColor{LightBackground}
\mymulticolumn{1}{x{8.4cm}}{Do use the prefix I for Interfaces.  Example: ITransfer} \tn 
% Row Count 19 (+ 2)
\hhline{>{\arrayrulecolor{DarkBackground}}-}
\end{tabularx}
\par\addvspace{1.3em}

\begin{tabularx}{8.4cm}{x{4 cm} x{4 cm} }
\SetRowColor{DarkBackground}
\mymulticolumn{2}{x{8.4cm}}{\bf\textcolor{white}{General Naming Conventions}}  \tn
% Row 0
\SetRowColor{LightBackground}
{\bf{Do}} & {\bf{Don't}} \tn 
% Row Count 1 (+ 1)
% Row 1
\SetRowColor{white}
Use easily readable identifier names. Favor readability over brevity. & Use underscore, hyphen or Hungarian notation. \tn 
% Row Count 5 (+ 4)
% Row 2
\SetRowColor{LightBackground}
Use semantically interesting generic names.  eg. GetAmountDue vs GetDecimalValue & Use identifiers that conflict with C\# \tn 
% Row Count 9 (+ 4)
% Row 3
\SetRowColor{white}
Use acronyms only if required, and only use widely accepted ones & Use abbreviations as part of the identifiers. \tn 
% Row Count 13 (+ 4)
\hhline{>{\arrayrulecolor{DarkBackground}}--}
\end{tabularx}
\par\addvspace{1.3em}

\begin{tabularx}{8.4cm}{X}
\SetRowColor{DarkBackground}
\mymulticolumn{1}{x{8.4cm}}{\bf\textcolor{white}{Types}}  \tn
% Row 0
\SetRowColor{LightBackground}
\mymulticolumn{1}{x{8.4cm}}{{\bf{Fields}}} \tn 
% Row Count 1 (+ 1)
% Row 1
\SetRowColor{white}
\mymulticolumn{1}{x{8.4cm}}{Typically nouns or noun phrases are used as names for the fields. e.g. \_salary} \tn 
% Row Count 3 (+ 2)
% Row 2
\SetRowColor{LightBackground}
\mymulticolumn{1}{x{8.4cm}}{{\bf{Properties}}} \tn 
% Row Count 4 (+ 1)
% Row 3
\SetRowColor{white}
\mymulticolumn{1}{x{8.4cm}}{Nouns, noun phrases or adjectives are used for naming properties} \tn 
% Row Count 6 (+ 2)
% Row 4
\SetRowColor{LightBackground}
\mymulticolumn{1}{x{8.4cm}}{Properties and Get methods should not be named alike.} \tn 
% Row Count 8 (+ 2)
% Row 5
\SetRowColor{white}
\mymulticolumn{1}{x{8.4cm}}{Boolean properties should be named with phrases like Is or Has.} \tn 
% Row Count 10 (+ 2)
% Row 6
\SetRowColor{LightBackground}
\mymulticolumn{1}{x{8.4cm}}{{\bf{Methods}}} \tn 
% Row Count 11 (+ 1)
% Row 7
\SetRowColor{white}
\mymulticolumn{1}{x{8.4cm}}{Typically verbs or verb phrases are used as names for the methods.  e.g. GetEncodingString()} \tn 
% Row Count 13 (+ 2)
% Row 8
\SetRowColor{LightBackground}
\mymulticolumn{1}{x{8.4cm}}{{\bf{Events}}} \tn 
% Row Count 14 (+ 1)
% Row 9
\SetRowColor{white}
\mymulticolumn{1}{x{8.4cm}}{Typically verbs or verb phrases are used as names for the events.} \tn 
% Row Count 16 (+ 2)
% Row 10
\SetRowColor{LightBackground}
\mymulticolumn{1}{x{8.4cm}}{In event handlers, use two parameter named sender and e.} \tn 
% Row Count 18 (+ 2)
% Row 11
\SetRowColor{white}
\mymulticolumn{1}{x{8.4cm}}{Concept of before and after should be given, e.g Closing, Closed, etc} \tn 
% Row Count 20 (+ 2)
\hhline{>{\arrayrulecolor{DarkBackground}}-}
\end{tabularx}
\par\addvspace{1.3em}

\begin{tabularx}{8.4cm}{X}
\SetRowColor{DarkBackground}
\mymulticolumn{1}{x{8.4cm}}{\bf\textcolor{white}{Enums}}  \tn
% Row 0
\SetRowColor{LightBackground}
\mymulticolumn{1}{x{8.4cm}}{Do not use prefixes or suffixes} \tn 
% Row Count 1 (+ 1)
% Row 1
\SetRowColor{white}
\mymulticolumn{1}{x{8.4cm}}{Usually names are plural nouns. E.g Teams, Colors} \tn 
% Row Count 2 (+ 1)
% Row 2
\SetRowColor{LightBackground}
\mymulticolumn{1}{x{8.4cm}}{Do not use flag as suffix for the names of flag enumerations} \tn 
% Row Count 4 (+ 2)
\hhline{>{\arrayrulecolor{DarkBackground}}-}
\end{tabularx}
\par\addvspace{1.3em}


% That's all folks
\end{multicols*}

\end{document}
